%% Copyright 2009 Jeffrey D. Hein



\documentclass{article}
\usepackage{tikz}   %TikZ is required for this to work.  Make sure this exists before the next line

\usepackage{tikz-3dplot}


\usepackage[active,tightpage]{preview}  %generates a tightly fitting border around the work
\PreviewEnvironment{tikzpicture}
\setlength\PreviewBorder{2mm}

\begin{document}


%set the plot display orientation
%synatax: \tdplotsetdisplay{\theta_d}{\phi_d}
\tdplotsetmaincoords{60}{110}

%define polar coordinates for some vector
%TODO: look into using 3d spherical coordinate system
\pgfmathsetmacro{\rvec}{.8}
\pgfmathsetmacro{\thetavec}{30}
\pgfmathsetmacro{\phivec}{60}

%start tikz picture, and use the tdplot_main_coords style to implement the display 
%coordinate transformation provided by 3dplot
\begin{tikzpicture}[scale=5,tdplot_main_coords]

%set up some coordinates 
%-----------------------
\coordinate (O) at (0,0,0);

%determine a coordinate (P) using (r,\theta,\phi) coordinates.  This command
%also determines (Pxy), (Pxz), and (Pyz): the xy-, xz-, and yz-projections
%of the point (P).
%syntax: \tdplotsetcoord{Coordinate name without parentheses}{r}{\theta}{\phi}
\tdplotsetcoord{P}{\rvec}{\thetavec}{\phivec}

%draw figure contents
%--------------------

%draw the main coordinate system axes
\draw[thick,->] (0,0,0) -- (1,0,0) node[anchor=north east]{$x$};
\draw[thick,->] (0,0,0) -- (0,1,0) node[anchor=north west]{$y$};
\draw[thick,->] (0,0,0) -- (0,0,1) node[anchor=south]{$z$};

%draw a vector from origin to point (P) 
\draw[-stealth,color=black] (O) -- (P) node[color = black, anchor = south west]{$P(r, \theta, \phi)$} node[pos=0.5, anchor = north]{$r$};

%draw projection on xy plane, and a connecting line
\draw[dashed, color=black] (O) -- (Pxy);
\draw[dashed, color=black] (P) -- (Pxy);

%draw the angle \phi, and label it
%syntax: \tdplotdrawarc[coordinate frame, draw options]{center point}{r}{angle}{label options}{label}
\tdplotdrawarc{(O)}{0.2}{0}{\phivec}{anchor=north}{$\phi$}

%set the rotated coordinate system so the x'-y' plane lies within the
%"theta plane" of the main coordinate system
%syntax: \tdplotsetthetaplanecoords{\phi}
\tdplotsetthetaplanecoords{\phivec}

%draw theta arc and label, using rotated coordinate system
\tdplotdrawarc[tdplot_rotated_coords]{(0,0,0)}{0.5}{0}{\thetavec}{anchor=south west}{$\theta$}

%draw some dashed arcs, demonstrating direct arc drawing
\draw[dashed,tdplot_rotated_coords] (\rvec,0,0) arc (0:90:\rvec);
\draw[dashed] (\rvec,0,0) arc (0:90:\rvec);

% %set the rotated coordinate definition within display using a translation
% %coordinate and Euler angles in the "z(\alpha)y(\beta)z(\gamma)" euler rotation convention
% %syntax: \tdplotsetrotatedcoords{\alpha}{\beta}{\gamma}
% \tdplotsetrotatedcoords{\phivec}{\thetavec}{0}

% %translate the rotated coordinate system
% %syntax: \tdplotsetrotatedcoordsorigin{point}
% \tdplotsetrotatedcoordsorigin{(P)}

% %use the tdplot_rotated_coords style to work in the rotated, translated coordinate frame
% \draw[thick,tdplot_rotated_coords,->] (0,0,0) -- (.5,0,0) node[anchor=north west]{$x'$};
% \draw[thick,tdplot_rotated_coords,->] (0,0,0) -- (0,.5,0) node[anchor=west]{$y'$};
% \draw[thick,tdplot_rotated_coords,->] (0,0,0) -- (0,0,.5) node[anchor=south]{$z'$};

%WARNING:  coordinates defined by the \coordinate command (eg. (O), (P), etc.)
%cannot be used in rotated coordinate frames.  Use only literal coordinates.  

% %draw some vector, and its projection, in the rotated coordinate frame
% \draw[-stealth,color=blue,tdplot_rotated_coords] (0,0,0) -- (.2,.2,.2);
% \draw[dashed,color=blue,tdplot_rotated_coords] (0,0,0) -- (.2,.2,0);
% \draw[dashed,color=blue,tdplot_rotated_coords] (.2,.2,0) -- (.2,.2,.2);

%show its phi arc and label
% \tdplotdrawarc[tdplot_rotated_coords,color=blue]{(0,0,0)}{0.2}{0}{45}{anchor=north west,color=black}{$\phi'$}

%change the rotated coordinate frame so that it lies in its theta plane.
%Note that this overwrites the original rotated coordinate frame
%syntax: \tdplotsetrotatedthetaplanecoords{\phi'}
% \tdplotsetrotatedthetaplanecoords{45}

%draw theta arc and label
% \tdplotdrawarc[tdplot_rotated_coords,color=blue]{(0,0,0)}{0.2}{0}{55}{anchor=south west,color=black}{$\theta'$}

\end{tikzpicture}

\end{document}


\end{tikzpicture}

% \begin{tikzpicture} % CENT

% %% some definitions

% \def\R{2.5} % sphere radius
% \def\angEl{35} % elevation angle
% \def\angAz{-105} % azimuth angle
% \def\angPhi{-40} % longitude of point P
% \def\angBeta{19} % latitude of point P

% %% working planes

% \pgfmathsetmacro\H{\R*cos(\angEl)} % distance to north pole
% \tikzset{xyplane/.estyle={cm={cos(\angAz),sin(\angAz)*sin(\angEl),-sin(\angAz),
%                               cos(\angAz)*sin(\angEl),(0,-\H)}}}
% \LongitudePlane[xzplane]{\angEl}{\angAz}
% \LongitudePlane[pzplane]{\angEl}{\angPhi}
% \LatitudePlane[equator]{\angEl}{0}

% %% draw xyplane and sphere
% \fill[ball color=white] (0,0) circle (\R); % 3D lighting effect
% \draw (0,0) circle (\R);

% %% characteristic points
% \coordinate (O) at (0,0);
% \coordinate[mark coordinate] (N) at (0,\H);
% \coordinate[mark coordinate] (S) at (0,-\H);
% \path[pzplane] (\angBeta:\R) coordinate[mark coordinate] (P);
% \path[pzplane] (\R,0) coordinate (PE);
% \path[xzplane] (\R,0) coordinate (XE);
% \path (PE) ++(0,-\H) coordinate (Paux); % to aid Phat calculation
% \coordinate[mark coordinate] (Phat) at (intersection cs: first line={(N)--(P)},
%                                         second line={(S)--(Paux)});

% % draw meridians and latitude circles

% \DrawLatitudeCircle[\R]{0} % equator
% %\DrawLatitudeCircle[\R]{\angBeta}
% \DrawLongitudeCircle[\R]{\angAz} % xzplane
% % \DrawLongitudeCircle[\R]{\angAz+90} % yzplane
% % \DrawLongitudeCircle[\R]{\angPhi} % pzplane

% %% draw xyz coordinate system

% \draw[xyplane,<->] (1.8*\R,0) node[below] {$x,\xi$} -- (0,0) -- (0,2.4*\R)
%     node[right] {$y,\eta$};
% \draw[->] (0,-\H) -- (0,1.6*\R) node[above] {$z,\zeta$};

% %% draw lines and put labels

% \draw[dashed] (P) -- (N) +(0.3ex,0.6ex) node[above left] {$\mathbf{N}$};
% \draw (P) -- (Phat) node[above right] {$\mathbf{\hat{P}}$};
% \path (S) +(0.4ex,-0.4ex) node[below] {$\mathbf{S}$};
% \draw[->] (O) -- (P) node[above right] {$\mathbf{P}$};
% \draw[dashed] (XE) -- (O) -- (PE);
% \draw[pzplane,->,thin] (0:0.5*\R) to[bend right=15]
%     node[pos=0.4,right] {$\beta$} (\angBeta:0.5*\R);
% \draw[equator,->,thin] (\angAz:0.4*\R) to[bend right=30]
%     node[pos=0.4,below] {$\phi$} (\angPhi:0.4*\R);
% \draw[thin,decorate,decoration={brace,raise=0.5pt,amplitude=1ex}] (N) -- (O)
%     node[midway,right=1ex] {$a$};

% \end{tikzpicture}

% \begin{tikzpicture} % MERC

% %% some definitions

% \def\R{3} % sphere radius
% \def\angEl{25} % elevation angle
% \def\angAz{-100} % azimuth angle
% \def\angPhiOne{-50} % longitude of point P
% \def\angPhiTwo{-35} % longitude of point Q
% \def\angBeta{33} % latitude of point P and Q

% %% working planes

% \pgfmathsetmacro\H{\R*cos(\angEl)} % distance to north pole
% \LongitudePlane[xzplane]{\angEl}{\angAz}
% \LongitudePlane[pzplane]{\angEl}{\angPhiOne}
% \LongitudePlane[qzplane]{\angEl}{\angPhiTwo}
% \LatitudePlane[equator]{\angEl}{0}

% %% draw background sphere

% \fill[ball color=white] (0,0) circle (\R); % 3D lighting effect
% %\fill[white] (0,0) circle (\R); % just a white circle
% \draw (0,0) circle (\R);

% %% characteristic points

% \coordinate (O) at (0,0);
% \coordinate[mark coordinate] (N) at (0,\H);
% \coordinate[mark coordinate] (S) at (0,-\H);
% \path[xzplane] (\R,0) coordinate (XE);
% \path[pzplane] (\angBeta:\R) coordinate (P);
% \path[pzplane] (\R,0) coordinate (PE);
% \path[qzplane] (\angBeta:\R) coordinate (Q);
% \path[qzplane] (\R,0) coordinate (QE);

% %% meridians and latitude circles

% % \DrawLongitudeCircle[\R]{\angAz} % xzplane
% % \DrawLongitudeCircle[\R]{\angAz+90} % yzplane
% \DrawLongitudeCircle[\R]{\angPhiOne} % pzplane
% \DrawLongitudeCircle[\R]{\angPhiTwo} % qzplane
% \DrawLatitudeCircle[\R]{\angBeta}
% \DrawLatitudeCircle[\R]{0} % equator

% % shifted equator in node with nested call to tikz 
% % (I didn't know it's possible)
% \node at (0,1.6*\R) { \tikz{\DrawLatitudeCircle[\R]{0}} };

% %% draw lines and put labels

% \draw (-\R,-\H) -- (-\R,2*\R) (\R,-\H) -- (\R,2*\R);
% \draw[->] (XE) -- +(0,2*\R) node[above] {$y$};
% \node[above=8pt] at (N) {$\mathbf{N}$};
% \node[below=8pt] at (S) {$\mathbf{S}$};
% \draw[->] (O) -- (P);
% \draw[dashed] (XE) -- (O) -- (PE);
% \draw[dashed] (O) -- (QE);
% \draw[pzplane,->,thin] (0:0.5*\R) to[bend right=15]
%     node[midway,right] {$\beta$} (\angBeta:0.5*\R);
% \path[pzplane] (0.5*\angBeta:\R) node[right] {$\hat{1}$};
% \path[qzplane] (0.5*\angBeta:\R) node[right] {$\hat{2}$};
% \draw[equator,->,thin] (\angAz:0.5*\R) to[bend right=30]
%     node[pos=0.4,above] {$\phi_1$} (\angPhiOne:0.5*\R);
% \draw[equator,->,thin] (\angAz:0.6*\R) to[bend right=35]
%     node[midway,below] {$\phi_2$} (\angPhiTwo:0.6*\R);
% \draw[equator,->] (-90:\R) arc (-90:-70:\R) node[below=0.3ex] {$x = a\phi$};
% \path[xzplane] (0:\R) node[below] {$\beta=0$};
% \path[xzplane] (\angBeta:\R) node[below left] {$\beta=\beta_0$};

% \end{tikzpicture}


% \begin{tikzpicture} % KART

% \def\R{2.5}

% \node[draw,minimum size=2cm*\R,inner sep=0,outer sep=0,circle] (C) at (0,0) {};
% \coordinate (O) at (0,0);
% \coordinate[mark coordinate] (Phat) at (20:2.5*\R);
% \coordinate (T1) at (tangent cs: node=C, point={(Phat)}, solution=1);
% \coordinate (T2) at (tangent cs: node=C, point={(Phat)}, solution=2);
% \coordinate[mark coordinate] (P) at ($(T1)!0.5!(T2)$);

% \draw[dashed] (T1) -- (O) -- (T2) -- (Phat) -- (T1) -- (T2);
% \draw[<->] (0,1.5*\R) node[above] {$y$} |- (2.5*\R,0) node[right] {$x$};
% \draw (O) node[below left] {$\mathbf{O}$} -- (P)
%     +(1ex,0) node[above=1ex] {$\mathbf{P}$};
% \draw (P) -- (Phat) node[above=1ex] {$\mathbf{\hat{P}}$};

% \end{tikzpicture}

\end{document} 