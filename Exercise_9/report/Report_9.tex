% arara: pdflatex
% !arara: biber
% !arara: pdflatex
% How to run: 
% 1) pdflatex "filename".tex
% 2) biber "filename"
% 3) pdflatex "filename".tex
% 4) pdflatex "filename".tex


\documentclass[x11names]{article}
\usepackage{verbatim}
\usepackage{listings}
\usepackage{graphicx}
\usepackage{color}
\usepackage{amsmath}
\usepackage{amssymb}
\usepackage[T1]{fontenc}
\usepackage{shadow}
\usepackage{hyperref}
\usepackage{physics}
\usepackage{url}
%For use in pictures
\usepackage{tikz}
\usepackage{tikz-3dplot}
\usepackage{wrapfig}

\usepackage[l2tabu]{nag} %NAgs when using bad practices


\usepackage{subcaption}
\usepackage[utf8]{inputenc}
\usepackage{booktabs} % Allows the use of \toprule, \midrule and \bottomrule in tables
\usepackage[font={small,it}]{caption}
\usepackage[margin=0.7in]{geometry} %Sets the margins in the document
\usepackage{siunitx}    %Allows use of SI units macros

%Defines calculator way to write powers of ten
\sisetup{output-exponent-marker=\textsc{e}}


% Change numbering and some commands
\renewcommand\thesection{Exercise \Roman{section}}
\renewcommand\thesubsection{\Roman{section}.\alph{subsection}}

%% references
\usepackage[style=authoryear,
            bibstyle=authoryear,
            backend=biber,
            % refsection=chapter,
            maxbibnames=99,
            maxnames=2,
            firstinits=true,
            uniquename=init,
            natbib=true,
            dashed=false]{biblatex}

\addbibresource{bibliography.bib}

\usepackage[capitalize]{cleveref}

\setcounter{tocdepth}{2}

\lstset{language=c++}
\lstset{alsolanguage=[90]Fortran}
\lstset{basicstyle=\small}
\lstset{backgroundcolor=\color{white}}
\lstset{frame=single}
\lstset{stringstyle=\ttfamily}
\lstset{keywordstyle=\color{red}\bfseries}
\lstset{commentstyle=\itshape\color{blue}}
\lstset{showspaces=false}
\lstset{showstringspaces=false}
\lstset{showtabs=false}
\lstset{breaklines}


\definecolor{keywords}{RGB}{255,0,90}
      \definecolor{comments}{RGB}{0,0,113}
      \definecolor{red}{RGB}{160,0,0}
      \definecolor{green}{RGB}{0,150,0}
       
      \lstset{language=Python, 
              basicstyle=\ttfamily\small, 
              keywordstyle=\color{keywords},
              commentstyle=\color{comments},
              stringstyle=\color{red},
              showstringspaces=false,
              identifierstyle=\color{green}
              }



\title{ Exercise 9 \\ Sommerjobb Numeriske Plasmaoppgaver }
\author{Gullik Vetvik Killie
		}

\renewcommand{\va}{\vec}

%%%%%%%%%%%%%%%%%%%%%%%%%%%%%%%%%%%%%%%%%%%%%%%%%%%%%%%%%%%%%%%%%%%%%%%%%%%%%%%%%%%%
% Actual text starts here
%%%%%%%%%%%%%%%%%%%%%%%%%%%%%%%%%%%%%%%%%%%%%%%%%%%%%%%%%%%%%%%%%%%%%%%%%%%%%%%%%%%%
\begin{document}


\maketitle

\section{}

\subsection{Theory}
  In this exercise we will investigate how electrically charged particles behave in the converging magnetic field above the north pole and the balance between the forces.
\\ \\
  A particle gyrating in will have a magnetic moment, dependant on it's perpendicular velocity, given by

  \begin{align}
    \va{\mu} &= \frac{mv_\perp^2}{2B} \va{e}_B
  \end{align}

  The gradient force felt by the particle is then given by

  \begin{align}
    \va{F}_B &= -\left( \va{\mu}\cdot\nabla \right)\va{B}
  \end{align}

  The particle will also feel a gravitational force and the force balance is given when these 2 forces are equal.

  \begin{align}
    -\va{F}_G + \va{F}_B &= 0  
    \\
    mg(z) &= \frac{mv_\perp^2}{2B(z)} \pdv{B_z(z)}{z}
  \end{align}
  \noindent If we let the magnetic field be approximated as in the previous exercises,

  \begin{align}
    \va{B}(\va{r}) &= \frac{\mu_0}{4\pi}\left( \frac{3\va{r}(\va{r}\cdot\va{m}  )}{r^5} - \frac{\va{m}}{r^3} \right)
    \\
    \va{g}(\va{r}) &= - \gamma \frac{M_e}{r^3}
      \begin{pmatrix}
        x \\ y \\ z
      \end{pmatrix}
  \end{align}

  \noindent if we then consider the case where the guiding center of the particle is \(200 \si{\kilo\meter}\) straight above the magnetic north, so the guiding center coordinates is
  \( \va{r}_0 = (0,0,200\si{\kilo\meter} + R_e)\). Then it becomes simpler to find the z directed part of the gradient.

  \begin{align}
    \pdv{B_z(z)}{z} &= \pdv{}{z}\left( \frac{\mu_0}{4\pi} \left(\frac{3z^2m_z}{z^5} - \frac{m_z}{z^3} \right)\right)
    \\
    &= \frac{\mu_0m_z}{4\pi} \left( - \frac{9}{z^4} + \frac{3}{z^4} \right)
    \\
    &= - \frac{3\mu_0m_z}{2\pi z^4}
  \end{align}

  Then the gyration velocity giving a balance in the force given by:

  \begin{align}
    v^2_\perp &=  2B(z)g(z) \left(\pdv{B_z(z)}{z} \right)^{-1}
  \end{align}
  Assuming that the particle is directly above the magnetic north pole, so the radius is given by the \(z\)-coordinate, the 3 factors become:
   \begin{align}
    \begin{split}
      B(z) &= \frac{\mu_0}{4\pi}\left( \frac{3z^2m_z}{z^5} - \frac{m_z}{z^3} \right) =  \frac{\mu_0 m_z}{2\pi z^3} \approx \frac{4\pi \times 10^{-7} \times 7.94 \times 10^{22}}{2\pi\times (6.571\times 10^6 \si{m})^3} \approx 5.597\times 10 ^{-5} \si{\tesla}
      \\
      g(z) &= - \gamma \frac{M_e}{z^3} z = -\frac{\gamma M_e}{z^2} \approx - \frac{6.7 \times 10^{-11} \times 5.9 \times 10^{24}}{(6.571\times 10^6 )^2} \si{\meter \per \second^2} \approx -9.1
      \\
      \pdv{B_z(z)}{z} &= - \frac{3\mu_0m_z}{2\pi z^4} \approx -\frac{6 \times 10^{-7}\times 7.94\times 10^{22}}{(6.571\times 10^6 )^4 } \approx 2.55\times 10^{-11}
    \end{split}
    \intertext{and the total initial perpendicular velocity will be}
    v^2_\perp &= 
  \end{align}
  

  % \begin{align}
  %  \pdv{B_z(z)}{z} &= \pdv{}{z}\left( \frac{3z^2m_z}{r^5} - \frac{m_z}{r^3} \right)
  %   \\
  %   \pdv{B_z(z)}{z} &= \frac{3 m_{z} \mu_{0} }{4 \pi } \left(\frac{3 z }{r^5} - \frac{5 z^{3}}{r^7}\right)
  % \end{align}
  % The gyration velocity where the force is balanced is then given by
  % \begin{align}
  %     mg(z) &= \frac{m_zv_\perp^2}{2B(z)}  \frac{3 m_{z} \mu_{0} }{4 \pi } \left(\frac{3 z }{r^5} - \frac{5 z^{3}}{r^7}\right)
  % \end{align}

\subsection{Results}




\appendix
\section{Rough estimate of the balancing velocity}
 

\section{Comments}
  \begin{itemize}
    \item In the text it's written that the particle should be placed \(\frac{1}{2}\) a gyro radius to the side of magnetic pole. I assume we want to place the guiding center of the particle at the magnetic pole, which would need the particle to be placed \(1\) gyro radius to the side.
    \item On the exercise 1: It should probably have some part describing what they should do with the solved equations of motions, e.g. plot it and comment on the trajectory of the particle.
    \item Parenthesis around the magnetic field, equation \(5\).
    \item In equation \(6\) the equation of motion is listed as 
    \[m\pdv{\va{v}}{t} = q\va{v} \cross \va{B} - m \va{g}\]
    , and in equation \(7\) the gravity vector is listed as negative with respect to the direction.
    One of these should be positive so the gravitational force works the right direction (give a negative \(v_z\)).
    \item In the hint section the Earths mass has the wrong number, it should be to the power \(24\) instead of \(23\)
  \end{itemize}

\section{Code}
  \label{sec:code}
  \lstinputlisting{../source/stability.py}


\end{document}