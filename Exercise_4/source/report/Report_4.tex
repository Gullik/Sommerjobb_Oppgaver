% arara: pdflatex
% !arara: biber
% !arara: pdflatex
% How to run: 
% 1) pdflatex "filename".tex
% 2) biber "filename"
% 3) pdflatex "filename".tex
% 4) pdflatex "filename".tex


\documentclass[x11names]{article}
\usepackage{verbatim}
\usepackage{listings}
\usepackage{graphicx}
\usepackage{a4wide}
\usepackage{color}
\usepackage{amsmath}
\usepackage{amssymb}
\usepackage[dvips]{epsfig}
\usepackage[T1]{fontenc}
% \usepackage{cite} % [2,3,4] --> [2--4]
\usepackage{shadow}
\usepackage{hyperref}
\usepackage{physics}
\usepackage{url}
\usepackage{tikz}
\usepackage{subcaption}
\usepackage[utf8]{inputenc}
\usepackage{booktabs} % Allows the use of \toprule, \midrule and \bottomrule in tables
\usepackage[font={small,it}]{caption}
\usepackage[margin=0.7in]{geometry} %Sets the margins in the document
\usepackage{siunitx}    %Allows use of SI units macros

%Defines calculator way to write powers of ten
\sisetup{output-exponent-marker=\textsc{e}}


% Change numbering and some commands
\renewcommand\thesection{Exercise \Roman{section}}
\renewcommand\thesubsection{\Roman{section}.\alph{subsection}}

%% references
\usepackage[style=authoryear,
            bibstyle=authoryear,
            backend=biber,
            % refsection=chapter,
            maxbibnames=99,
            maxnames=2,
            firstinits=true,
            uniquename=init,
            natbib=true,
            dashed=false]{biblatex}

\addbibresource{bibliography.bib}
% \addbibresource{top.bib}

% \bibliography{bibliography}
% \bibliography{top}


\usepackage[capitalize]{cleveref}

\setcounter{tocdepth}{2}

\lstset{language=c++}
\lstset{alsolanguage=[90]Fortran}
\lstset{basicstyle=\small}
\lstset{backgroundcolor=\color{white}}
\lstset{frame=single}
\lstset{stringstyle=\ttfamily}
\lstset{keywordstyle=\color{red}\bfseries}
\lstset{commentstyle=\itshape\color{blue}}
\lstset{showspaces=false}
\lstset{showstringspaces=false}
\lstset{showtabs=false}
\lstset{breaklines}


\definecolor{keywords}{RGB}{255,0,90}
      \definecolor{comments}{RGB}{0,0,113}
      \definecolor{red}{RGB}{160,0,0}
      \definecolor{green}{RGB}{0,150,0}
       
      \lstset{language=Python, 
              basicstyle=\ttfamily\small, 
              keywordstyle=\color{keywords},
              commentstyle=\color{comments},
              stringstyle=\color{red},
              showstringspaces=false,
              identifierstyle=\color{green}
              }



\title{ Exercise 3 \\ Sommerjobb Numeriske Plasmaoppgaver }
\author{Gullik Vetvik Killie
		}


%%%%%%%%%%%%%%%%%%%%%%%%%%%%%%%%%%%%%%%%%%%%%%%%%%%%%%%%%%%%%%%%%%%%%%%%%%%%%%%%%%%%
% Actual text starts here
%%%%%%%%%%%%%%%%%%%%%%%%%%%%%%%%%%%%%%%%%%%%%%%%%%%%%%%%%%%%%%%%%%%%%%%%%%%%%%%%%%%%
\begin{document}


\maketitle

\section{Exercise}

\subsection{Theory}
  Starting from Faraday's law, Ohm's, Àmpere's law and the equation of motion for a infinitely conducting cold plasma, see \cref{tab:symbols} for explanation of the symbols,

  \begin{align}
    \nabla \cross \va{E} =& \pdv{\va{b}}{t} \label{eq:Faraday}
    \\
    \va{E} &= -\pdv{\xi}{t} \cross \va{B}_0 \label{eq:Ohm}
    \\
    \nabla \cross \va{b}  &= \mu_0\va{j}  \label{eq:Ampere}
    \\
    \rho \pdv[2]{\xi}{t} &= \va{j} \cross \va{B}_0 \label{eq:EOM}
  \end{align}

  \begin{table}
            \centering
            \begin{tabular}{| c | c | c |}
                 Symbol & Meaning & Value
                 \\
                 $\va{E}$ & Electric Field  & -
                 \\
                 $\va{B}_0$ & External Magnetic Field & \((4.0\times10^{11}x^{-3}, 0, 0) \si{\tesla}\)
                 \\
                 $\va{b}$ & Magnetic Field (Plasma) & -
                 \\
                 $\va{j}$ & Current Density & -
                 \\
                 $\va{\xi}$ & Plasma Displacement & -
                 \\
                 $\rho $ & Mass Density & \((50x^{-4}, 0, 0) \si{\tesla}\)
                 \\
                 $ x $  & Length  \\ -
                 $\mu_0 $ & Vacuum Permeability & $ 4\pi10^{-7} \si{\newton\per\ampere} $
                 \\
                 $\va{v}_A$ &  Alfvèn Velocity & $ \va{B}_0/\sqrt{\mu_0\rho}$ 
                 \\
                 $\nu$  & Plasma Displacement Velocity  & -
                 \\
                 $R_e$ & Earth Radius & $6.371\times 10^3 \si{\meter}$
            \end{tabular}
            \caption{Overview over the symbols, and values used}
            \label{tab:symbols}
      \end{table}

    Combining \cref{eq:Faraday} and \cref{eq:Ohm}, and integrating we get
    \begin{align}
      \nabla \cross \left(-\xi \cross \va{B}_0\right) &= \va{b}
      \intertext{Crossing the equation with $\nabla$ and inserting \cref{eq:Ampere}}
      \nabla \cross \nabla \cross \left(-\xi \cross \va{B}_0\right) &= \mu_0 \va{j}
      \intertext{Inserting this into \cref{eq:EOM} } 
      \pdv[2]{\xi}{t} &= \frac{1}{\mu_0\rho} \left(\nabla \cross \nabla \cross \left(-\xi \cross \va{B}_0\right) \right) \cross \va{B}_0
      \intertext{Crossing the with \(\va{B}_0\) and swapping the order of cross products we obtain}
      \pdv[2]{\va{B}_0 \cross \va{\xi}}{t} &= \va{v}_A \cross \va{v}_A \left(\nabla \cross \nabla \cross \left(\va{B}_0 \cross \va{\xi}\right) \right)
    \end{align}

    Assuming a shear and sinusoidal wave along the y-axis, \(\va{\xi} = e^{i\omega t} \va{e}_y \) this simplifies to a one dimensional wave equation with a variable cooefficient

    \begin{align}
      \pdv{\xi(x)}{x^2} &= -\xi(x) \frac{\omega}{v^2_A(x)}
      \intertext{Introducing \( \nu(x) = \pdv{\xi(x)}{t} \) we get two first order coupled differential equations}
      \pdv{\xi(x)}{t} &= \nu(x)
      \\
      \pdv{\nu(x)} &= -\xi(x)\frac{\omega^2}{v^2_A(x)}
      \intertext{Using a forward finite step we obtain}
      \xi(x + h) &= \xi(x) + h\nu(x)
      \\
      \nu(x + h) &= \nu(x) - h \frac{\omega^2}{v^2_A(x)} \xi(x)
    \end{align}

    We want to search for the standing waves on the domain \(x = [ -5R_E, 5R_E ]\), 

\subsection{Results}

      

\appendix
\section{Comments regarding the exercise}
      \begin{itemize}
            \item
      \end{itemize}


\section{Code}
      

\end{document}