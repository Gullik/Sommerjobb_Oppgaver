% arara: pdflatex
% !arara: biber
% !arara: pdflatex
% How to run: 
% 1) pdflatex "filename".tex
% 2) biber "filename"
% 3) pdflatex "filename".tex
% 4) pdflatex "filename".tex


\documentclass[x11names]{article}
\usepackage{verbatim}
\usepackage{listings}
\usepackage{graphicx}
\usepackage{a4wide}
\usepackage{color}
\usepackage{amsmath}
\usepackage{amssymb}
\usepackage[dvips]{epsfig}
\usepackage[T1]{fontenc}
% \usepackage{cite} % [2,3,4] --> [2--4]
\usepackage{shadow}
\usepackage{hyperref}
\usepackage{physics}
\usepackage{url}
\usepackage{tikz}
\usepackage{subcaption}
\usepackage[utf8]{inputenc}
\usepackage{booktabs} % Allows the use of \toprule, \midrule and \bottomrule in tables
\usepackage[font={small,it}]{caption}
\usepackage[margin=0.7in]{geometry} %Sets the margins in the document
\usepackage{siunitx}    %Allows use of SI units macros

%Defines calculator way to write powers of ten
\sisetup{output-exponent-marker=\textsc{e}}


% Change numbering and some commands
\renewcommand\thesection{Exercise \Roman{section}}
\renewcommand\thesubsection{\Roman{section}.\alph{subsection}}

%% references
\usepackage[style=authoryear,
            bibstyle=authoryear,
            backend=biber,
            % refsection=chapter,
            maxbibnames=99,
            maxnames=2,
            firstinits=true,
            uniquename=init,
            natbib=true,
            dashed=false]{biblatex}

\addbibresource{bibliography.bib}
% \addbibresource{top.bib}

% \bibliography{bibliography}
% \bibliography{top}


\usepackage[capitalize]{cleveref}

\setcounter{tocdepth}{2}

\lstset{language=c++}
\lstset{alsolanguage=[90]Fortran}
\lstset{basicstyle=\small}
\lstset{backgroundcolor=\color{white}}
\lstset{frame=single}
\lstset{stringstyle=\ttfamily}
\lstset{keywordstyle=\color{red}\bfseries}
\lstset{commentstyle=\itshape\color{blue}}
\lstset{showspaces=false}
\lstset{showstringspaces=false}
\lstset{showtabs=false}
\lstset{breaklines}


\definecolor{keywords}{RGB}{255,0,90}
      \definecolor{comments}{RGB}{0,0,113}
      \definecolor{red}{RGB}{160,0,0}
      \definecolor{green}{RGB}{0,150,0}
       
      \lstset{language=Python, 
              basicstyle=\ttfamily\small, 
              keywordstyle=\color{keywords},
              commentstyle=\color{comments},
              stringstyle=\color{red},
              showstringspaces=false,
              identifierstyle=\color{green}
              }



\title{ Exercise 3 \\ Sommerjobb Numeriske Plasmaoppgaver }
\author{Gullik Vetvik Killie
		}


%%%%%%%%%%%%%%%%%%%%%%%%%%%%%%%%%%%%%%%%%%%%%%%%%%%%%%%%%%%%%%%%%%%%%%%%%%%%%%%%%%%%
% Actual text starts here
%%%%%%%%%%%%%%%%%%%%%%%%%%%%%%%%%%%%%%%%%%%%%%%%%%%%%%%%%%%%%%%%%%%%%%%%%%%%%%%%%%%%
\begin{document}


\maketitle

\section{Exercise}

\subsection{Theory}

In this text we will simulate an oxygen ion, \(O^+\) and an electron, \(e\) in the dipole field of the earth. We will use Euler's method to get the discretization of the position and work in Cartesian coordinates. From the simulation we will plot the trajectory the particles will follow, and also examine how the kinetic energy of the the particles evolve as they move along the magnetic field lines.

An approximation of Earth's dipole magnetic field is
\begin{align}
      \va{B}(\va{r}) &= \frac{\mu_0}{4\pi}\frac{3\va{r}\left( \va{m} \cdot \va{r}\right)}{r^5} - \frac{\va{m}}{r^3}
\end{align}
where \( \va{r} \), \(\mu_0\) and \(\va{m}\) is the position, vacuum permeability and magnetic moment respectively.

The equation of motion for a charged particle is

\begin{align}
      \pdv{\va{v}}{t} &= \frac{q}{m}\left( \va{v} \cross \va{B}(\va{r}) \right)
      \intertext{with}
      \pdv{\va{r}}{t} &= \va{v}
\end{align}

Using Euler's method to dicretize it we arrive at the following two equations

\begin{align}
      \va{v}(t+h) &= \va{v}(t) + h\frac{q}{b} \left( \va{v}(t) \cross \va{B}(\va{r}(t)) \right)
      \\
      \va{r}(t+h) &= \va{r}(t) + h \va{v}(t)
\end{align}

Then we use the same algorithm as in the earlier exercises to move forward in time

 \begin{enumerate}
            \item Declare variables necessary variables.
            \item Set initial conditions
            \item For-loop over timesteps. 
                  \begin{itemize}
                        \item Update \( \va{v}(t+h) \) and \( \va{r}( t+h ) \).
                  \end{itemize}
            \item Plot and analyze results
\end{enumerate}

\subsubsection{Rounding Errors}
      In this exercise we need to consider a few rounding errors, due to distances and velocities at a very different value being important for the physics involved.
      \begin{itemize}
            \item The magnetic field is dependent on the total distance from the center of the Earth, which the particle is located, which is of the order \(\order{4}\)
            \item The particle's gyration radius is of the order \(\order{-13}\)
            \item Depending on the timestep chosen the particle's change in position will be even lower than the gyration radius, on the order \(\order{-17}\)
            \item Using float 64 numbers the small change in particle position will be rounded away, when the two numbers are added together.
            \item This problem is solved by shifting the coordinate system so it is centered around the particle gyration, and instead adding the extra distance into the calculation of the magnetic field
      \end{itemize}

      For the treatment of the velocity we cannot as easily just shift the coordinate system, since the movement along the z-axis is important for the curvature of the magnetic field.
      \begin{itemize}
            \item The order of \(dv_z\) is \( \order{-26} \)
            \item The order of \(v_z\) is \( \order{3} \)
      \end{itemize}

      To solve this we need one solution for the velocity and one solution for the influence the velocity has on the position, for the velocity a shift of the velocity coordinate system is enough.
      



\subsection{Results}

      

\appendix
\section{Comments regarding the exercise}
      \begin{itemize}
            \item I assume you mean the perpendicular and parallel \textit{kinetic} energy in exercise 2.
            \item Could it be better to plot it first, and afterwards find the energy. I think it is quite natural to plot it while working on it.
            \item I suppose we are supposed to find that the total kinetic energy is conserved (in a static magnetic field, \(\pdv{\va{B}}{t} = 0\) ), and grad-B drift should be visible (and the Alfven approximations, that the magnetic field variations is small compared to gyration frequency and radius, below hold
            \[ \left|\frac{1}{\omega_c} \pdv{\ln{B}}{t}\right| << 1 \] 
            \[ \left| \frac{(\rho_c \cdot \nabla)\va{B}}{B} \right| << 1 \] 
            \item A curvature drift should be there as well?
            \item I had hoped to see the magnetic mirror effect as well
            \item The movement along the x-axis is so small compared to the magnitude of the coordinate so it got rounded away, had to shift the coordinate system \(5R_e\) to keep it
      \end{itemize}


\section{Code}
      

\end{document}